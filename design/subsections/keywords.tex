\subsection{Identifiers, keywords and types}
SDIR introduces only one additional identifier with the symbol \codeword{\$} (see Section \ref{section:symbol}).\\
Every op and region specific to SDIR is prefixed by the keyword \codeword{sdir} followed by a dot to clearly mark the end of the prefix. Some examples are: \codeword{sdir.tasklet\{\}}, \codeword{sdir.return}, \codeword{sdir.alloc()}. \\
SDIR contains only one additional type on top of the builtin types: \codeword{!sdir.memlet<>}. See Section \ref{section:memlet}. 

\subsubsection{Symbols} \label{section:symbol}
The op \codeword{sdir.alloc_symbol} creates a new symbol, but does not return anything. In order to access symbols the identifier \codeword{\$()} can be used. See Figure \ref{fig:symbol}. One can describe arithmetic expressions inside the identifier. Example:\codeword{\$(2*N - 1)}. Parentheses can be omitted for a single symbol. \codeword{\$N} is the same as \codeword{\$(N)}.
\smc{symbol}{Symbol used for range in map}

 \subsubsection{Symbolic sizes}
 Symbols (see Section \ref{section:symbol}) can be used to define the size of an \codeword{sdir.memlet}. One simply replaces the constants in the type by the symbol identifier \codeword{\$()}.\\
 Example:\codeword{\%A = sdir.alloc() : !sdir.memlet<\$(2*N)xi32>}.
 
 \subsubsection{Write conflict resolution}
In some cases an op might write to a location that contains the result of an other op. To resolve this conflict any op that writes to a location has an optional attribute called \codeword{wcr}. Examples:\codeword{sdir.copy\{wcr="add"\} \%a -> \%c} or \codeword{sdir.store\{wcr="max"\} \%1, \%a : i32}. "overwrite" is the default if no \codeword{wcr} is provided. A function may be provided as well. The function must have the signature \codeword{old, new -> resolved} and the types must match the memlet types. See Figure \ref{fig:wcr}.
\codeblock{wcr}{Write conflict resolution with custom function}

\subsubsection{Multiple basic blocks and branching}
Tasklets (see Section \ref{section:tasklet}) and \codeword{sdir.func} are the only regions that may have multiple basic blocks and branching. Every other region has only one basic block and therefore terminator operations can be omitted. SDFG regions (see Section \ref{section:multistate}) contain branches, but these are provided as attributes and therefore there is no need for branching with \codeword{sdir.br} or \codeword{sdir.cond_br}.

\subsubsection{Unnamed outputs}
Any output that is unnamed will be auto-generated with the prefix \codeword{__} (double underscore). Therefore explicitly naming any output with the same prefix is strictly forbidden for the user.